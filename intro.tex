\section{Introduction}\label{sec:intro}

\fixme{introduction here}

IPC mechanisms enable multiple processes in a system to communicate with each other. Different methods can be used for this purpose, depending on the application requirements and the process runtime environment. 
Pipes
Pipes provide a unidirectional communication channel between 2 processes. The data written on the write end of the pipe is buffered by the kernel until it is read from the other end of the pipe.
Sockets
< Enter brief description here>
Shared Memory
Shared memory provides the communicating processes, a simultaneous access to a section of memory. Synchronization primitives are used to coordinate access to the shared region. 

\if 0
\begin{figure}
  \includegraphics[width=\linewidth]{figs/first-fig.pdf}
  \caption{Energy/Performance for BERET/C-Cores/Revolver/Composite Cores (bigLITTLE) \newline 
           \textnormal{\small (See section~\ref{sec:methodology} for methodology)} }
  \label{fig:existing}
\end{figure}
\fi


\paragraph{Contributions}
\begin{itemize}
\item \textbf {Contributions here},
\end{itemize}

\paragraph{Paper Organization}
first ($\S$\ref{sec:overview}), and subsequently describe the
evalaution ($\S$\ref{sec:eval}) and results($\S$\ref{sec:res}).
Finally conclusion($\S$\ref{sec:conc}).


